\documentclass[aspectratio=169,xcolor=table]{beamer}
\usepackage{algorithm}
\usepackage{algpseudocode}
\usepackage[utf8]{inputenc}
\usepackage[T1]{fontenc}
\usepackage{lipsum, lmodern}
\usepackage{csquotes}
\usepackage{xcolor}
\usepackage[portuguese]{babel}
\usetheme{DCC}
\graphicspath{{imgs/}}

\author{Antoniel Magalhães, João Leahy, Luis Felipe}
\title{Problema de Colocação Ótima de Câmeras de Segurança}
\institute{Universidade Federal da Bahia}
\date{\today}

\begin{document}

% Slide inicial: Título
\begin{frame}[plain,noframenumbering]
    \titlepage
\end{frame}

% Agenda
\begin{frame}{Agenda}
    \tableofcontents
\end{frame}

\section{Introdução}

% Introdução
\begin{frame}{Introdução}
    \begin{itemize}
        \item Contextualização e Motivação
        \item Importância da Teoria dos Grafos na Segurança
    \end{itemize}
\end{frame}

\section{Descrição do Problema}

% Descrição do Problema
\begin{frame}{Descrição do Problema}
    \begin{itemize}
        \item Formalização do Problema
        \item Modelagem em Grafos
        \item Restrições e Função Objetivo
    \end{itemize}
\end{frame}

\section{Solução Algorítmica}

% Solução Algorítmica
\begin{frame}{Solução Algorítmica}
    \begin{itemize}
        \item Abordagem Heurística: Algoritmo Guloso para Cobertura de Vértices
        \item Visão Geral do Pseudo-Código
        \item Complexidade e Aplicação em Teoria dos Grafos
    \end{itemize}
\end{frame}

\section{Detalhes de Implementação}

% Detalhes de Implementação
\begin{frame}{Detalhes de Implementação}
    \begin{itemize}
        \item Estrutura do Código e Design Orientado a Objetos
        \item Estruturas de Dados: Listas de Adjacência
        \item Técnicas de Otimização
    \end{itemize}
\end{frame}

\section{Experimentos}

% Experimentos
\begin{frame}{Experimentos}
    \begin{itemize}
        \item Metodologia e Instâncias
        \item Parâmetros e Critérios de Teste
    \end{itemize}
\end{frame}

\section{Resultados}

% Resultados
\begin{frame}{Resultados}
    \begin{itemize}
        \item Gráficos e Tabelas dos Resultados
        \item Avaliação da Abordagem Heurística
    \end{itemize}
\end{frame}

\section{Conclusão}

% Conclusão
\begin{frame}{Conclusão}
    \begin{itemize}
        \item Aplicabilidade da Teoria dos Grafos
        \item Eficiência e Praticidade da Solução
    \end{itemize}
\end{frame}

\section{Referências}

% Referências
\begin{frame}{Referências}
    \begin{itemize}
        \item Lopes Filho, J. G. (2019). Problema do Caixeiro Viajante com Coleta Opcional de Bônus, Tempo de Coleta e Passageiros. Tese de Doutorado, Universidade Federal do Rio Grande do Norte, Natal-RN.
        \item Carvalho, M. R. (2022). Métodos Heurísticos para o TSP-OBP. Journal of Combinatorial Optimization.
        \item Carnielli, W. and Epstein, R. (2017). Computabilidade e Funções Computáveis. UNESP.
        \item Goldbarg, M. and Goldbarg, E. (2012). Grafos: Conceitos, algoritmos e aplicações. Elsevier.
    \end{itemize}
\end{frame}

\end{document}