%  #region preâmbulo
\documentclass[12pt, a4paper]{report}
\usepackage[top=3cm,left=3cm,right=2cm,bottom=2cm]{geometry}
\linespread{1.3}
\setlength{\parindent}{1.25cm}
\usepackage{indentfirst}
\usepackage[utf8]{inputenc}
\usepackage[brazil]{babel}
\usepackage{amsmath}
\usepackage{amsthm}
\usepackage{amsfonts}
\usepackage{amssymb}
\usepackage{graphicx}
\usepackage{color}
\usepackage{multicol}
\usepackage[normalem]{ulem}
\usepackage{wrapfig}
\usepackage{caption}
\usepackage{fancybox}
\usepackage[pdfstartview=FitH]{hyperref}
\usepackage{subfigure}
\bibliographystyle{plain}
\usepackage{algorithm}
\usepackage{algpseudocode}
\usepackage{float}


\graphicspath{{Figuras/}}

\renewcommand{\theenumii}{\alph{enumii}}
\DeclareMathOperator{\sen}{sen}
\DeclareMathOperator{\tg}{tg}
\DeclareMathOperator{\arctg}{arctg}
\DeclareMathOperator{\cotg}{cotg}
\DeclareMathOperator{\agm}{agm}

\newtheorem{thm}{Teorema}[section]
\newtheorem{dfn}{Definição}[section]
\newtheorem{prob}{Problema}[section]
\newtheorem{cor}{Corolário}[section]
\newtheorem{prop}{Proposição}[section]
\newtheorem{lem}{Lema} [section]

\newcounter{contar}
%  #endregion preâmbulo

% #region Variáveis 
\newcommand{\nomeUniversidade}{Universidade Federal da Bahia}
\newcommand{\nomeInstituto}{Instituto de Computação}
\newcommand{\nomeCurso}{MATA53 - Teoria dos grafos}
\newcommand{\nomeProfessor}{Islame Felipe da Costa Fernandes}
\newcommand{\nomeGrupo}{\sc{\large{Antoniel Magalhães}} \\
\sc{\large{João Leahy}} \\
\sc{\large{Luis Felipe}}}
\newcommand{\titulo}{\sc{\Large{Problema de Colocação Ótima de Câmeras de Segurança no bairro da Ondina}}}
% #endregion Variáveis 

\begin{document}

% #region capa
\pagestyle{empty}
\begin{center}
\includegraphics[height=2.5cm]{UFBA.jpg}
\hspace{2cm}
\end{center}

\begin{center}
\sc{\large{\nomeUniversidade}} \\
\sc{\large{\nomeInstituto}} \\
\sc{\small{\nomeCurso}} \\

\vspace{4cm}

\titulo

\vspace{4.5cm}

\nomeGrupo


\vspace{5.5cm}

\textbf{Salvador - Bahia} \\
\today
\end{center}
% #endregion capa

% #region folha de rosto
\newpage
\begin{center}
\titulo

\vspace{4cm}

\nomeGrupo
\end{center}

\vspace{4cm}

\begin{flushright}
\begin{minipage}{8.6cm}
Projeto final entregue ao professor \nomeProfessor\ 
como método avaliativo da disciplina \nomeCurso


\end{minipage}
\end{flushright}
 
\vspace{8cm}


\begin{center}
\textbf{Salvador - Bahia} \\
\today
\end{center}

% #endregion folha de rosto

% #region Índice
\newpage
\tableofcontents
\thispagestyle{empty}
\newpage
\setcounter{page}{1}
\pagestyle{plain}
% #endregion Índice


\section*{Introdução}

\subsection*{Contextualização e Motivação}
A teoria dos grafos oferece um poderoso conjunto de ferramentas matemáticas para modelar e resolver problemas complexos de otimização em redes. No contexto da segurança pública, um problema particularmente relevante é a otimização do posicionamento de câmeras de vigilância. O bairro de Ondina, em Salvador, apresenta um cenário ideal para aplicação desses conceitos, por concentrar pontos estratégicos como a Universidade Federal da Bahia, estabelecimentos comerciais, hotéis e áreas residenciais, além de um intenso fluxo turístico devido às suas praias. A modelagem deste cenário através de grafos permite uma abordagem sistemática para maximizar a cobertura de vigilância com recursos limitados.

\subsection*{Justificativa}
A aplicação de conceitos fundamentais da teoria dos grafos, como cobertura de vértices, dominação e problemas de localização de facilidades, fornece uma base teórica sólida para abordar o problema de posicionamento de câmeras. Este trabalho permite explorar na prática diversos algoritmos e técnicas estudados na disciplina MATA53 - Teoria dos Grafos, como algoritmos gulosos, programação dinâmica e métodos de otimização em grafos. A escolha do bairro de Ondina como objeto de estudo possibilita uma aplicação real desses conceitos, contribuindo tanto para o aprendizado acadêmico quanto para uma possível solução prática de segurança pública.

\subsection*{Objetivo Geral e Específicos}
O objetivo geral deste trabalho é aplicar conceitos e algoritmos da teoria dos grafos para desenvolver uma solução que otimize o posicionamento de câmeras de segurança no bairro de Ondina, Salvador.

Os objetivos específicos incluem:
\begin{itemize}
    \item Modelar a região de Ondina como um grafo, onde vértices representam possíveis localizações de câmeras e arestas representam conexões visuais ou físicas entre pontos;
    \item Implementar e comparar diferentes algoritmos estudados na disciplina para resolver o problema de cobertura mínima;
    \item Desenvolver uma solução que considere restrições práticas como orçamento e áreas prioritárias;
    \item Analisar a complexidade computacional e eficiência dos algoritmos implementados;
    \item Avaliar a aplicabilidade das soluções teóricas em um cenário real de implementação.
\end{itemize}

\subsection*{Metodologia}
TODO

\subsection*{Organização do Trabalho}
TODO

\chapter{Trabalhos Correlatos}
TODO - Analisar artigos relacionados ao problema de cobertura ótima, localização de facilidades e câmeras de segurança.

\chapter{Descrição Formal do Problema}

\section{Formalização}
TODO - Definir o problema matematicamente, considerando um grafo \(G = (V, E)\), onde os vértices representam locais possíveis para câmeras e as arestas representam conexões entre pontos.

\section{Restrições do Problema}
TODO - Descrever as restrições, como orçamento limitado, número máximo de câmeras, áreas prioritárias, etc.

\section{Função Objetivo}
TODO - Definir a função que deve ser otimizada, como maximizar a cobertura de áreas críticas ou minimizar o custo total.

\section{Modelagem em Grafos}
TODO - Explicar como o problema pode ser representado como um problema em teoria dos grafos, como problemas de cobertura de vértices ou problemas de localização de facilidades.

\chapter{Solução Algorítmica}

\section{Pseudo-Código}
TODO - Apresentar um algoritmo de alto nível, como um GRASP ou uma abordagem baseada em programação linear.

\section{Detalhes de Implementação}
\subsection{Classes, Diagramas, Estruturas}
TODO - Detalhar a estrutura do código, como organização das classes e métodos.

\chapter{Experimentos}

\section{Metodologia}
\subsection{Instâncias}
TODO - Descrever os dados utilizados, como mapas do bairro da Ondina e pontos potenciais para câmeras.

\subsection{Parâmetros}
TODO - Explicar os parâmetros ajustados nos experimentos, como alcance das câmeras, custo, etc.

\subsection{Testes e Critérios de Análise}
TODO - Detalhar como os resultados serão avaliados (tempo de execução, cobertura alcançada, etc.).

\section{Resultados}
TODO - Apresentar os resultados dos experimentos em gráficos ou tabelas.

\chapter{Considerações Finais}

\section{Conclusão}
TODO - Resumir as contribuições e os resultados obtidos.

%-------------Bibliografia------------------
\newpage
\renewcommand{\refname}{Referências Bibliográficas}
\addcontentsline{toc}{chapter}{Referências Bibliográficas}
\bibliography{Bibliografia}
\nocite{lopesfilho2019, carvalho2022, carnielli}

\end{document}